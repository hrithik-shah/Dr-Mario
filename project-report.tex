\documentclass{article}

%% Page Margins %%
\usepackage{geometry}
\geometry{
    top = 0.75in,
    bottom = 0.75in,
    right = 0.75in,
    left = 0.75in,
}

\usepackage{amsmath}
\usepackage{graphicx}
\usepackage{parskip}

\title{Assembly Project: Dr Mario}

% TODO: Enter your name
\author{Hrithik Parag Shah}

\begin{document}
\maketitle

\section{Instruction and Summary}

\begin{enumerate}

    \item Which milestones were implemented? 

        All milestones have been completed.

        \textbf{Easy Features:}
        \begin{itemize}
            \item Gravity (1)
            \item Increase gravity speed (2)
            \item Game over and retry (4)
            \item Preview next capsule (11)
            \item Display highest score so far (10)
            \item Make viruses disappear (14)
        \end{itemize}

        \textbf{Hard Features:}
        \begin{itemize}
            \item Score board (1)
        \end{itemize}

    \item How to view the game:
    
    \begin{enumerate}

    \item - Unit width in pixels: 8
    \item - Unit height in pixels: 8
    \item - Display width in pixels: 256
    \item - Display height in pixels: 256
    \item - Base Address for Display: 0x10008000 (\$gp)

    \end{enumerate}

    \item Data:
    
    \begin{verbatim}
# Memory address for display
ADDR_DSPL:              .word 0x10008000        

# Array in memory that has a copy of the pixels inside the bottle
INNER_BOTTLE:           .word 0:408             

# 2 element array storing the 2 addresses where the capsule should be drawn
CAPSULE:                .word 0x100081ac:2      

# 2 element array storing the 2 colors for the current capsule in play
COLOR_CAPSULE:          .space 8                

# Stores the memory address where the preview of the next capsule is drawn
NEXT_CAPSULE:           .word 0x100081e4        

# Stores the 2 colors of the next capsule
NEXT_CAPSULE_COLOR:     .space 8                

# Stores the memory address of the keyboard input
ADDR_KBRD:              .word 0xffff0000        

# Initial memory address where to draw a new capsule
CENTR_BTTL:             .word 0x10008124

# Stores the memory address where the top left of the bottle is
TOP_LEFT_BOTTLE:        .word 0x10008304

# Red color
RED:                    .word 0x00ff0000

# Yellow color
YELLOW:                 .word 0x00ffff00

# Blue color
BLUE:                   .word 0x000000ff

# Color for red virus
LIGHT_PINK:             .word 0x00ff80a0

# Color for yellow virus
ORANGE:                 .word 0x00ff9000

# Color for blue virus
LIGHT_BLUE:             .word 0x008888ff

# Memory address where to draw score
SCORE_POSITION:         .word 0x10008cd0

# Stores the current score
SCORE:                  .word 0x0

# Stores the number of times the game loop has run to control falling speed
GAME_LOOP_COUNT:        .word 0x0

# Memory address where to draw Dr Mario image
DR_MARIO_POSITION:  .word 0x10008350

# Memory address where to draw red virus (under Dr Mario)
RED_VIRUS_POSITION:     .word 0x100087dc

# Memory address where to draw yellow virus (under Dr Mario)
YELLOW_VIRUS_POSITION:  .word 0x10008864

# Memory address where to draw blue virus (under Dr Mario)
BLUE_VIRUS_POSITION:    .word 0x100087ec

# Number of game loops to wait before move capsule down
FALLING_SPEED:          .word 0xf

# Memory address where to draw high score
HIGH_SCORE_POSITION:    .word 0x10008950

# Stores the current high score
HIGH_SCORE:             .word 0x0
    \end{verbatim}

\item Game Summary:
% TODO: Tell us a little about your game.
\begin{itemize}
\item   Cool game that you can play with w, a, s, and d keys
\item   You can exit anytime with the q key
\item   When game is over, the r key can be pressed to start a new game
\item   WARNING: Game can be quite addictive
\end{itemize}

    
\end{enumerate}

\section{Attribution Table}
% TODO: If you worked in partners, tell us who was 
%       responsible for which features. Some reweighting 
%       might be possible in cases where one group member
%       deserves extra credit for the work they put in.

\begin{center}
\begin{tabular}{|| c | c ||}
\hline
 Student 1 (Name and student number) &  Student 2 (Name and student number) \\ 
 \hline
 Task & Task\\
 \hline
 Task & Task\\
 \hline
 Task & Task\\ 
 \hline
 Task & Task\\ 
 \hline
 Task & Task\\
 \hline
 Task & Task\\  
 \hline
\end{tabular}
\end{center}

% TODO: Fill out the remainder of the document as you see 
%       fit, including as much detail as you think 
%       necessary to better understand your code. 
%       You can add extra sections and subsections to 
%       help us understand why you deserve marks for 
%       features that were more challenging than they
%       might initially seem.


\end{document}